\documentclass[manual.tex]{subfiles} 
\begin{document}


%%%%%%%%%%%%%%%%%%%%%%%%%%%%%%%%%%%%%%%%%%%%%%%%%%%%%%%%%%%%%%%%%%%%
\subsection{Invocation}
%%%%%%%%%%%%%%%%%%%%%%%%%%%%%%%%%%%%%%%%%%%%%%%%%%%%%%%%%%%%%%%%%%%%

The algorithm must be invokable via the system command-line using the following command with arguments:

\texttt{<algo\_executable> <instance\_name> <instance\_specific\_information>
<cutoff\_time> <cutoff\_length> <seed> <param>} \texttt{<param>} \texttt{<param>}...

\begin{description}
\item [{algo\_executable}] Exactly what is specified in the \textbf{algo} argument in the scenario file.
 
\item [{instance\_name}] The name of the problem instance we are executing
against.

\item [{instance\_specific\_information}] An arbitrary string associated
with this instance as specified in the \textbf{instance\_file }. If
no information is present then a ``0'' is always passed here. 

\item [{cutoff\_time}] The amount of time in seconds that the target algorithm
is permitted to run. It is the responsibility of the callee
to ensure that this is obeyed. It is not necessary that that the actual
algorithm execution time (wall clock time) be below this value (\eg{If the algorithm needs to cleanup, or it's only possible to terminate the algorithm at certain stages}).

\item [{cutoff\_length}] A domain specific measure of when the algorithm should consider itself done.

\item [{seed}] A positive integer that the algorithm should use to seed
itself (for reproducibility). ``-1'' is used when the algorithm is \textbf{deterministic}

\item [{param}] A setting of an active parameter for the selected configuration
as specified in the Algorithm Parameter File. EALib will only pass
parameters that are active. Additionally EALib is not guaranteed
to pass the parameters in any particular order. The exact format for
each parameter is:\\
\texttt{-name~'value'}

\end{description}

All of the arguments above will always be passed, even if they are inapplicable, in which case a dummy value will be passed.

%%%%%%%%%%%%%%%%%%%%%%%%%%%%%%%%%%%%%%%%%%%%%%%%%%%%%%%%%%%%%%%%%%%%
\subsection{Output}\label{sec:wrapper_output}
%%%%%%%%%%%%%%%%%%%%%%%%%%%%%%%%%%%%%%%%%%%%%%%%%%%%%%%%%%%%%%%%%%%%

The Target Algorithm is free to output anything, which will be ignored
but must at some point output a line (only once) in the following
format%
\footnote{ParamILS in not a typo. While other values are possible including
SMAC, HAL. ParamILS is probably the most portable. The exact Regex
that is used in this version is: \textbf{\^\textbackslash{}s{*}(Final)?\textbackslash{}s{*}{[}Rr{]}esult\textbackslash{}s+(?:(for)|(of))\textbackslash{}s+(?:(HAL)|(ParamILS)|(SMAC)|(this
wrapper))}}:%
\\
\\
\texttt{Result for ParamILS: <solved>, <runtime>, <runlength>, <quality>, <seed>, <additional rundata>} 
\begin{description}

\item [{solved}] Must be one of \textbf{SAT} (signifying a successful run that was satisfiable), \textbf{UNSAT} (signifying a successful run that was unsatisfiable), \textbf{TIMEOUT} if the algorithm didn't finish within the allotted time, \textbf{CRASHED} if something untoward happened during the algorithm run, or \textbf{ABORT} if something prevents the target algorithm for successfully executing and it is believed that further attempts would be futile. 

EALib also supports reporting \textbf{SATISFIABLE} for \textbf{SAT} and \textbf{UNSATISFIABLE} for \textbf{UNSAT}.
\\ \textsc{Note:} These are only aliases and EALib will not preserve which alias was used in the log or state files.

\textbf{ABORT} can be useful in
cases where the target algorithm cannot find required files, or a
permission problem prevents access to them. This will cause EALib to
stop running immediately. Use this option with care, it should only be reported when the algorithm knows for \textsc{certain} that subsequent results may fail. For things like sporadic network failures, and other cosmic-ray induced failures, one should consider using \textbf{CRASHED} in combination with the \textbf{-$\!~$-retry-crashed-count}  and \textbf{-$~\!\!$-abort-on-crash} options, to mitigate these.

In other files or the log you may see the following following additional types used. \textbf{RUNNING} which signifies a result that is currently in the middle of being run, and \textbf{KILLED} which signifies that EALib internally decided to terminate the run before it finished. These are internal values only, and wrappers are NOT permitted to output these values. If these values are reported by the wrapper, it will be treated as if the run had status \textbf{CRASHED}.

\item [{runtime}] The amount of CPU time used during this algorithm run.
EALib does not measure the CPU time directly, and this is the amount
that is used with respect to \textbf{tunerTimeout}. You may get
unexpected performance degradation when this amount is heavily under
reported \footnote{This typically happens when targeting very short algorithm
runs with large overheads that aren't accounted for.}. 

\textsc{Note:}The \textbf{runtime }should always be strictly less
than the requested \textbf{cutoff\_time } when reporting \textbf{SAT
}or \textbf{UNSAT}. The runtime must be strictly greater than zero (and not NaN).

\item [{runlength}] A domain specific measure of how far the algorithm
progressed. This value must be from the set: ${-1} \cup [0,+\infty]$.

\item [{quality}] A domain specific measure of the quality of the solution. This value needs to be 
from the set: $(-\infty, +\infty)$. 

\textsc{Note}: In certain cases, such as when using log transforms in the model, this value must be: $(0, +\infty)$.

\item [{seed}] The seed value that was used in this target algorithm execution.
\textsc{Note:} This seed \textsc{Must} match the seed that the algorithm was
called with. This is used as fail-safe check to ensure that the output
we are parsing really matches the call we requested.
\item[{additional rundata}] A string (not containing commas, or newline characters) that will be saved with the run.
\textsc{Note}:\textbf{additional rundata} is not compatible with ParamILS at time of writing, and so wrappers should not include this or the preceding comma if they wish to be compatible.


\end{description}
All fields except for \textbf{additional rundata} are mandatory. If the field is not applicable for your scenario a 0 can be substituted.
\end{document}